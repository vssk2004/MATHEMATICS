\section{Variables}


\subsection{Variables}

\slide{Variables}{
    In mathematics, a variable is used as a placeholder when referring to something that:
    \begin{enumerate}
        \item has one or more possible values, but we do not know what they are
        \item is meant to be equally true for all elements in a given set, and so you don't want to be
        restricted to considering only a particular value.
    \end{enumerate}

    It is important to keep in mind that variables do not need to represent only numbers.
    In linear algebra, they can be matrices or vectors. In probability, they can be random
    variables. In essence, their uses are varied and widespread.
}


\subsection{Usage of First Type}

\slide{Variables}{
    For its first use, consider asking the following question:
    \begin{center}
        Is there a number with the following property: doubling it and adding
        3 gives the same result as squaring it?
    \end{center}

    Here, we introduce a variable \(x\) to avoid ambiguity:
    \begin{center}
        Is there a number \(x\) with the property that \(2x+3=x^2\)?
    \end{center}
}


\subsection{Usage of Second Type}

\slide{Variables}{
    For its second use, consider the following statement:
    \begin{center}
        No matter what odd integer we choose, if we square it and then divide
        the result by 8, we'll get a remainder of 1.
    \end{center}

    Here, we introduce a variable \(n\) to avoid ambiguity:
    \begin{center}
        No matter what odd integer \(n\) we choose, \(n^2\) has a remainder of 1
        when we divide it by 8.
    \end{center}
}


\subsection{Example}

\slide{Example}{
    Rewrite the following sentences using variables.
    \begin{enumerate}
        \item[(a)] Are there integers with the property that the sum of their squares is equal to the square of their sum?
        \item[(b)] The fourth power of every real number is positive.
        \item[(c)] There is a real number whose natural logarithm is $-3$.
    \end{enumerate}
}


\slide{Solution to Example}{
    \begin{enumerate}
    \item[(a)] Are there integers \(m\) and \(n\) such that \(m^2 + n^2 = (m+n)^2\)?
    \item[(b)] If \(x\) is a real number, then \(x^4\) is positive.
    \item[(c)] There exists a real number \(y\) such that \(\ln(y) = -3\).
    \end{enumerate}
}



\section{Important Kinds of Mathematical Statements}


\subsection{Overview of the Kinds of Mathematical Statements}

\slide{Universal, Conditional, and Existential Statements}{
    \dfn{A \textbf{universal statement} says that a certain property is true for all elements in a set.
    For example: All positive numbers are greater than zero.}

    \dfn{A \textbf{conditional statement} says that if one thing is true then some other thing also has to be true.
    For example: If 378 is divisible by 18, then 378 is divisible by 6.}

    \dfn{An \textbf{existential statement} says that there is at least one thing for which a particular property is true.
    For example: There is a prime number that is even.}
}


\slide{Universal, Conditional, and Existential Statements}{
    Here are some more examples of the different kinds of statements.
    \begin{itemize}
        \item Every $2 \times 2$ matrix $C$ has an inverse $C^{-1}$. \\
        (Universal Statement)
        \item If $3$ and $4$ are elements of a set $W$, then $\{3,4\}$ is a subset of $W$. \\
        (Conditional Statement)
        \item There is a function defined on the set of all real numbers whose derivative is 
        \(
            \frac{x}{\sqrt{x^2 + 9}}.
        \) \\
        (Existential Statement)
    \end{itemize}
}


\subsection{Universal Conditional Statements}

\slide{Universal Conditional Statements}{
    Universal statements contain some variation of the words “for every”.

    Conditional statements contain versions of the words “if-then.”

    A \textbf{universal conditional} statement is a statement that is both universal and conditional.
    Consider the following example:
    \begin{center}
        For all animals \(a\), if \(a\) is a dog, then \(a\) is a mammal.
    \end{center}
}


\slide{Hiding Universality or Conditionality}{
    Consider the statement:
    \begin{center}
        For all animals \(a\), if \(a\) is a dog, then \(a\) is a mammal.
    \end{center}

    It can be rewritten in a that makes it appear purely universal or purely conditional.
    For example, the statement can be rewritten to make its conditional nature explicit but
    its universal nature implicit:
    \begin{center}
        If \(a\) is a dog, then \(a\) is a mammal.
    \end{center}

    The statement can also be expressed so that its universal nature is explicit and its
    conditional nature is implicit:
    \begin{center}
        For every dog \(a\), \(a\) is a mammal.
    \end{center}
}


\subsection{Universal Existential Statements}

\slide{Universal Existential Statements}{
    A \textbf{universal existential} statement has two parts. The first part is universal because
    it says that a certain property is true for all objects of a given type. The second part is
    existential because it asserts the existence of something.

    For example:
    \begin{center}
        Every real number has an additive inverse.
    \end{center}
}


\slide{Rewriting Universal Existential Statements}{
    Needless to say, universal existential statements can be rewritten, just like universal
    conditional statements. The statement:
    \begin{center}
        Every real number has an additive inverse.
    \end{center}
    can be rewritten as:
    \begin{itemize}
        \item All real numbers have additive inverses.
        \item For every real number \(r\), there is an additive inverse for \(r\).
        \item For every real number \(r\), there exists a real number \(s\) such that \(s\) is an
        additive inverse for \(r\).
    \end{itemize}
}


\subsection{Existential Universal Statements}

\slide{Existential Universal Statements}{
    An \textbf{existential universal} statement has two parts. The first is existential because it
    asserts that a certain object exists. The second part is universal because the object satisfies
    a certain property for all things of a certain kind.

    For example:
    \begin{center}
        There exists a positive integer that is less than or equal to every positive integer.
    \end{center}
}


\slide{Rewriting Existential Universal Statements}{
    The statement:
    \begin{center}
        There exists a positive integer that is less than or equal to every positive integer.
    \end{center}
    can be rewritten as:
    \begin{itemize}
        \item Some positive integer is less than or equal to every positive integer.
        \item There is a positive integer \(m\) such that every positive integer is
        greater than or equal to \(m\).
        \item There is a positive integer \(m\) with the property that for every
        positive integer \(n\), \(m \leq n\).
    \end{itemize}
}