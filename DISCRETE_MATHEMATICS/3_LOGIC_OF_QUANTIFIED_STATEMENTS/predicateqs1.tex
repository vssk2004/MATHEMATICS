\section{Predicates and Quantified Statements I}


\subsection{Predicates}

\slide{Proposition vs. Predicate}{
    Consider the following:
    \begin{enumerate}
        \item[(1)] \(5 + 5 = 11\)
        \item[(2)] \(x^2 = 16\)
    \end{enumerate}

    In the list above, \((1)\) is a proposition, but \((2)\) is not. Why?

    It's because \((2)\) may be either true or false depending on the value of \(x\).
}


\slide{Predicate}{
    In grammar, `predicate' refers to the part of a sentence that gives information
    about the subject.
    
    The predicate is the part of the sentence from which the subject has been removed.
    In logic, predicates generally have two parts.

    Consider the statement ``\(x\) is greater than 3.'' The first part, the variable \(x\), is the subject
    of the statement. The second part—the predicate, “is greater than 3”—refers to a property that
    the subject of the statement can have.

    We can denote the statement “\(x\) is greater than 3” by \(P(x)\), where \(P\) denotes the
    predicate “is greater than 3” and \(x\) is the variable. Once a value has been assigned to the
    variable \(x\), the statement \(P(x)\) (also called the propositional function) becomes a
    proposition and has a truth value.
}


\slide{Predicate}{
    \dfn{A \textbf{predicate} is a sentence that contains a finite number of variables and becomes
    a proposition when specific values are substituted for the variables.}
    
    The \emph{domain} of a predicate variable is the set of all values that may be substituted in place of the variable.

    The \emph{truth set} of a predicate, \(P(x)\), is the set of all elements in its domain that make \(P(x)\) true
    when they are substituted for \(x\).
}


\slide{Example}{
    Let \(A(c,n)\) denote the statement ``Computer \(c\) is connected to network \(n\),'' 
    where \(c\) is a variable representing a computer and \(n\) is a variable 
    representing a network. Suppose that the computer MATH1 is connected 
    to network CAMPUS2, but not to network CAMPUS1. What are the values of 
    \(A(\text{MATH1}, \text{CAMPUS1})\) and \(A(\text{MATH1}, \text{CAMPUS2})\)?

    \underline{Solution}:\\[0.5em]
    \(A(\text{MATH1}, \text{CAMPUS1}) = \text{False}\)\\
    \(A(\text{MATH1}, \text{CAMPUS2}) = \text{True}\).
}


\subsection{Quantifiers}

\slide{Quantifiers}{
    A propositional function (predicate) \(P(x)\) is not a proposition until it has a truth value.
    Until now, we could convert a propositional function into a proposition by assigning a value to the variable.

    Now, we will convert a propositional function into a proposition using \textbf{quantifiers}.

    The two most widely used quantifiers are:
    \begin{enumerate}
        \item[(1)] Universal Quantifier
        \item[(2)] Existential Quantifier
    \end{enumerate}
}


\slide{Universal Quantifier}{
    The symbol \(\forall\) is called the universal quantifier. Depending on the context, it is read as
    either “for every” or “for each” or “for all.”

    For example, another way to express the sentence “All human beings are mortal” is to write:
    \[\forall \text{ humans } x, x \text{ is mortal.}\]

    If you let \(H\) be the set of all human beings, then you can symbolize the statement more formally by writing:
    \[\forall x \in H, x \text{ is mortal.}\]
}


\slide{Universal Quantifier}{
    \dfn{Let \(P(x)\) be a predicate and let \(D\) be the domain of the variable \(x\). The \textbf{universal quantification} of \(P(x)\) is the statement
    \[\forall x \in D, \ P(x).\]
    This tells us that the proposition \(P(x)\) must be true \emph{for all} values of \(x\) in the domain \(D\).}
}


\slide{Example 1}{
    Let \(P(x)\) be the statement \(x+1 > x\). What is the truth value of the quantification \(\forall x \in \mathbb{R}, P(x)\)?

    \underline{Solution}:\\[0.5em]
    The predicate \(P(x)\) is \(x+1 > x\). If \(x\) is subtracted from both sides, this is equivalent to \(1 > 0\).
    Needless to say, \(1 > 0\) is true \(\forall x \in \mathbb{R}\).

    Therefore, truth value of the quantification \(\forall x \in \mathbb{R}, P(x)\) is true.
}


\slide{Example 2}{
    Let \(Q(x)\) be the statement \(x < 2\). What is the truth value of the quantification \(\forall x \in \mathbb{R}, Q(x)\)?

    \underline{Solution}:\\[0.5em]
    The predicate \(Q(x)\) is \(x < 2\). The domain of \(x\) is the set of all real numbers.
    If the domain is \(\mathbb{R}\), then there exists at least one real number \(x\)
    whose value is less than 2.

    A \emph{counterexample} to the quantification \(\forall x \in \mathbb{R}, Q(x)\) is \(x = 0\).

    Therefore, truth value of the quantification \(\forall x \in \mathbb{R}, Q(x)\) is false.
}


\slide{Existential Quantifier}{
    The symbol \(\exists\) denotes “there exists” and is called the existential quantifier.
    For example, the sentence “There is a student in CS50” can be written as:
    \[\exists \text{ a person } p \text{ such that } p \text{ is a student in CS50.}\]

    If we let \(P\) be the set of all people, then we can symbolize the statement more formally by writing:
    \[\exists p \in P \text{ such that } p \text{ is a student in CS50.}\]
}


\slide{Existential Quantifier}{
    \dfn{Let \(P(x)\) be a predicate and let \(D\) be the domain of the variable \(x\). The \textbf{existential
    quantification} of \(P(x)\) is the statement
    \[\exists x \in D \text{ such that } P(x).\]
    This tells us that the proposition \(P(x)\) is true for at least one value of \(x\) in the domain \(D\).}
}


\slide{Example 1}{
    Let \(P(x)\) denote the statement \(x > 3\). What is the truth value of the quantification \(\exists x \in \mathbb{R}, P(x)\)?

    \underline{Solution:}\\[0.5em]
    The predicate \(P(x)\) is \(x > 3\). We need to check if there exists at least one number \(x\) in the set of all real numbers
    that satisfies the predicate.

    Needless to say, an \(x \in \mathbb{R}\) does exist that satisfies \(x > 3\), like \(x = 4\).

    Therefore, the truth value of the quantification \(\exists x \in \mathbb{R}, P(x)\) is true.
}


\slide{Example 2}{
    Let \(Q(x)\) denote the statement \(x = x+1\). What is the truth value of the quantification \(\exists x \in \mathbb{R}, Q(x)\)?

    \underline{Solution:}\\[0.5em]
    The predicate \(Q(x)\) is \(x = x+1\). If \(x\) is subtracted from both sides, we have \(0 = 1\).
    This is not true \(\forall x \in \mathbb{R}\).

    Therefore, the truth value of the quantification \(\exists x \in \mathbb{R}, Q(x)\) is false.
}


\slide{Quantifiers over Finite Domains}{
    When the domain of a quantifier is finite, that is, when all its elements can be listed, quantified
    statements can be expressed using propositional logic.

    In particular, when the elements of the domain \(D\) are \(x_1, x_2, \dots, x_n\), the universal quantification \(\forall x \in D, \ P(x)\) is the
    same as the conjunction: \(P(x_1) \land P(x_2) \land \dots \land P(x_n)\).

    Similarly, when the elements of the domain \(D\) are \(x_1, x_2, \dots, x_n\), the existential quantification \(\exists x \in D \text{ such that } Q(x)\)
    is the same as the disjunction: \(P(x_1) \lor P(x_2) \lor \dots \lor P(x_n)\).
}


\slide{Summary of Quantifiers}{
\[\resizebox{1\textwidth}{!}{
$\begin{array}{|c|c|}
\hline
\textbf{Universal Quantifier} & \textbf{Existential Quantifier} \\
\hline
\forall & \exists \\
\hline
\text{``For all''} & \text{``There exists''} \\
\hline
\textit{When true?} &
\textit{When true?} \\
\text{When } P(x) \text{ is true for every } x \text{ in the domain} &
\text{There is an } x \text{ in the domain for which } P(x) \text{ is true} \\
\hline
\textit{When false?} &
\textit{When false?} \\
\text{There is an } x \text{ in the domain for which } P(x) \text{ is false} &
\text{When } P(x) \text{ is false for every } x \text{ in the domain} \\
\hline
\forall x\, P(x) \equiv P(x_1)\land P(x_2)\land \cdots \land P(x_n) &
\exists x\, P(x) \equiv P(x_1)\lor P(x_2)\lor \cdots \lor P(x_n) \\
\hline
\end{array}$}
\]
}


\subsection{Universal Conditional Statements}

\slide{Universal Conditional Statements}{
    One of the most important form of mathematical statements is the \textbf{universal
    conditional statement}. It has the form:
    \[\forall x, \text{ if } P(x) \text{ then } Q(x).\]

    More formally, it is written:
    \[\forall x, P(x) \to Q(x).\]
    
    It is common to omit explicit identification of the domain of predicate variables in
    universal conditional statements.
}


\slide{Examples}{
    Rewrite the following statements more formally:
    \begin{enumerate}
        \item[(1)] If a real number is an integer, then it is a rational number.
        \item[(2)] All bytes have eight bits.
    \end{enumerate}

    \underline{Solution}:\\[0.5em]
    \begin{enumerate}
        \item[(1)] \(\forall x, \text{ if } x \in \mathbb{Z} \text{, then } x \in \mathbb{Q}.\)
        \item[(2)] \(\forall b, \text{ if } b \text{ is a byte, then } b \text{ has 8 bits.}\)
    \end{enumerate}
}


\subsection{Implicit Quantification}

\slide{Equivalent Universal Statements}{
    Consider the statement:
    \[\text{All squares are rectangles.}\]

    Its universal quantification is:
    \[\forall \text{ squares } x, \; x \text{ is a rectangle.}\]

    This can also be written as a universal conditional statement:
    \[\forall x, \text{ if } x \text{ is a square, then } x \text{ is a rectangle.}\]
}


\slide{Implicit Quantification}{
    Consider the statement:
    \[\text{If a number is an integer, then it is a rational number.}\]

    This statement is an example of \emph{implicit} universal quantification.
    Now, we can convert this conditional statement into a universal statement.

    \[\forall x, \text{ if } x \in \mathbb{Z} \text{ then } x \in \mathbb{Q}.\]

    Mathematicians use a double arrow to indicate implicit quantification:
    \[x \in \mathbb{Z} \; \Rightarrow \; x \in \mathbb{Q}.\]
}


\slide{Implicit Quantification}{
    \dfn{Let \(P(x)\) and \(Q(x)\) be predicates and suppose the common domain of \(x\) is \(D\).
    \begin{itemize}
        \item The notation \(P(x) \Rightarrow Q(x)\) means that every element in the truth set of \(P(x)\) is in the truth set of \(Q(x)\), or, equivalently, \(\forall x,\; P(x) \to Q(x)\).
        
        \item The notation \(P(x) \Leftrightarrow Q(x)\) means that \(P(x)\) and \(Q(x)\) have identical truth sets, or, equivalently, \(\forall x,\; P(x) \leftrightarrow Q(x)\).
    \end{itemize}}
}


\slide{Example}{
Let
\[
\begin{aligned}
Q(n) &\text{ be ``}n \text{ is a factor of } 8,\text{''} \\
R(n) &\text{ be ``}n \text{ is a factor of } 4,\text{''} \\
S(n) &\text{ be ``}n < 5 \text{ and } n \ne 3.\text{''}
\end{aligned}
\]

Suppose the domain of \(n\) is \(\mathbb{Z}^+\). 
Use the \(\Rightarrow\) and \(\Leftrightarrow\) symbols to indicate true relationships among \(Q(n)\), \(R(n)\), and \(S(n)\).

\underline{Solution}:\\[0.5em]
\begin{itemize}
    \item \(R(n) \Rightarrow Q(n)\).
    \item \(S(n) \Rightarrow Q(n)\).
    \item \(R(n) \Leftrightarrow S(n)\).
\end{itemize}
}