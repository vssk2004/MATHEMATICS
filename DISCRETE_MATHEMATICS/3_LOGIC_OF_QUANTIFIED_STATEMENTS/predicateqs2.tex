\section{Predicates and Quantified Statements II}


\subsection{Negations of Quantified Statements}

\slide{Negation of Universal Quantifier}{
    Consider the proposition ``All roads lead to Rome.'' Many people would say that the negation is
    ``No road leads to Rome.'' However, this is not true.

    If even one road does not lead to Rome, then the sweeping statement that \emph{all} roads
    lead to Rome is false.

    Therefore, a correct negation is ``It is not the case that all roads lead to Rome.'' Equivalently,
    we can also say that ``There is a road that does not lead to Rome.''
    Formally, the negation is ``There exists at least one road that does not lead to Rome.''
}


\slide{Negation of Universal Quantifier}{
    \thm{The negation of a statement of the form
    \[\forall x \in D, P(x)\]
    is logically equivalent to a statement of the form
    \[\exists x \in D \text{ such that } \neg P(x).\]
    Symbolically,
    \[\neg\left[\forall x \in D, P(x)\right] \equiv \exists x \in D \text{ such that } \neg P(x).\]
    }
}


\slide{Negation of Existential Quantifier}{
    Consider the proposition ``A student in this class has taken Calculus.''
    One might be tempted to say that the negation is ``All students in the class have taken
    Calculus.'' However, this is not true.

    If no student in the class has taken Calculus, then the sweeping statement that \emph{at least one}
    student has taken Calculus is false.

    Therefore, a correct negation is ``It is not the case that
    a student in this class has taken Calculus.'' Equivalently, we say that ``No student
    in this class has taken Calculus.''
    Formally, the negation is ``All students in the class have not taken Calculus.''
}


\slide{Negation of Existential Quantifier}{
    \thm{The negation of a statement of the form
    \[\exists x \in D \text{ such that } Q(x)\]
    is logically equivalent to a statement of the form
    \[\forall x \in D, \neg Q(x).\]
    Symbolically,
    \[\neg \left[\exists x \in D \text{ such that } Q(x)\right] \equiv \forall x \in D, \neg Q(x).\]
    }
}


\slide{Examples}{
    Rewrite the following statements formally. Then write their negations.
    \begin{enumerate}
        \item No politicians are honest.
        \item The number 1,357 is not divisible by any integer between 1 and 37.
    \end{enumerate}

    \underline{Solution}:\\[0.5em]
    \begin{enumerate}
        \item No politicians are honest.\\
        \(\Rightarrow\) For all politicians \(p\), there is no honest \(p\).\\
        \(\Rightarrow \forall p \in P, \text{ there is no honest } p.\)\\
        Negation is: \(\exists p \in P \text{ such that } p \text{ is honest.}\)

        \item The number 1,357 is not divisible by any integer between 1 and 37.\\
        \(\Rightarrow \forall x \in \{1, 2, 3, \dots, 37\}, \text{ 1,357 is not divisible by } x.\)\\
        Negation is: \(\exists x \in \{1, 2, 3, \dots, 37\} \text{ such that 1,357 is divisible by } x.\)
    \end{enumerate}
}


\slide{Negations Made Easy!}{
    An easier way to remember the negation of a quantified statement is to think of the negation
    like an operator (which it is) such as \(-\).

    For instance, \(-(5-2) = -5+2 = -3\). Similarly, we apply the distributive property
    when we see the negation operator.

    \[\neg \left[\forall x \ P(x)\right] = \exists x \ \neg P(x)\]
    \[\neg \left[\forall x \ \neg P(x)\right] = \exists x \ P(x)\]
    \[\neg \left[\exists x \ Q(x)\right] = \forall x \ \neg Q(x)\]
    \[\neg \left[\exists x \ \neg R(x)\right] = \forall x \ R(x)\]
}


\subsection{Negations of Universal Conditional Statements}

\slide{Negation of Universal Conditional Statements}{
    Recall that a conditional proposition \(p \to q\) can be rewritten as \(\neg p \lor q\).
    Therefore, we can find the negation of a universally quantified conditional statement
    relatively easily.

    \[\neg \left[\forall x, P(x) \to Q(x)\right] \equiv \exists x \text{ such that } \neg \left(P(x) \to Q(x)\right)\]
    \[\neg \left(P(x) \to Q(x)\right) \equiv \neg \left(\neg P(x) \lor Q(x)\right) \equiv P(x) \land \neg Q(x)\]

    \[\neg \left[\forall x, P(x) \to Q(x)\right] \equiv \exists x \text{ such that } \left(P(x) \land \neg Q(x)\right)\]
}


\subsection{Relation among \texorpdfstring{\(\forall\), \(\exists\), \(\land\), and \(\lor\)}{}}

\slide{Relation among \texorpdfstring{\(\forall\), \(\exists\), \(\land\), and \(\lor\)}{}}{
    Suppose that the domain of \(x\) is \(D = \{x_1, x_2, \dots, x_n\}\)
    \[\forall x \in D, P(x) \equiv P(x_1) \land P(x_2) \land \dots \land P(x_n).\]
    \[\exists x \text{ such that } Q(x) \equiv Q(x_1) \lor Q(x_2) \lor \dots \lor Q(x_n).\]
}


\subsection{Vacuous Truth of Universal Statements}

\slide{Vacuous Truth of Universal Statements}{
    Consider a room with no balloons. Now, consider the statement:
    \begin{center}
        ``All the balloons in the room are blue.''
    \end{center}
    Is this statement true or false? The statement is false if, and only if, its negation is true.
    Its negation is:
    \begin{center}
        ``There exists a balloon in the room that is not blue.''
    \end{center}
    The only way this negation can be true is for there to actually be a non-blue balloon in the
    room, and there is not! Hence the negation is false, and so the statement is true ``by default.''
}


\slide{Vacuous Truth of Universal Statements}{
    In general, a statement of the form
    \[\forall x \in D, \text{ if } P(x) \text{ then } Q(x)\]
    is called \textbf{vacuously true} or \textbf{true by default} if, and only if, \(P(x)\) is false for every \(x\) in \(D\).

    In mathematics, the phrase ``in general'' signals that what is to follow is a generalization of
    an example that always holds true.
}


\subsection{Variants of Universal Conditional Statements}

\slide{Variants of Universal Conditional Statements}{
    Consider a statement of the form \(\forall x \in D\), if \(P(x)\) then \(Q(x)\).

    \begin{enumerate}
        \item Its \textbf{contrapositive} is the statement 
        \(\forall x \in D\), if \(\sim Q(x)\) then \(\sim P(x)\).
        \item Its \textbf{converse} is the statement 
        \(\forall x \in D\), if \(Q(x)\) then \(P(x)\).
        \item Its \textbf{inverse} is the statement 
        \(\forall x \in D\), if \(\sim P(x)\) then \(\sim Q(x)\).
    \end{enumerate}

    A universal conditional statement is logically equivalent to its contrapositive.
}


\subsection{Necessary and Sufficient Conditions, Only If}

\slide{Necessary and Sufficient Conditions, Only If}{
``\(\forall x\), \(r(x)\) is a \textbf{sufficient condition} for \(s(x)\)'' 
means \\ ``\(\forall x\), if \(r(x)\) then \(s(x)\).''

``\(\forall x\), \(r(x)\) is a \textbf{necessary condition} for \(s(x)\)'' 
means \\ ``\(\forall x\), if \(\sim r(x)\) then \(\sim s(x)\)'' or, equivalently, 
``\(\forall x\), if \(s(x)\) then \(r(x)\).''

``\(\forall x\), \(r(x)\) \textbf{only if} \(s(x)\)'' 
means \\ ``\(\forall x\), if \(\sim s(x)\) then \(\sim r(x)\)'' or, equivalently, 
``\(\forall x\), if \(r(x)\) then \(s(x)\).''
}