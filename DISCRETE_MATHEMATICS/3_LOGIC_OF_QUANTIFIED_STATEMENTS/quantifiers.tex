\section{Multiple or Nested Quantifiers}


\subsection{Interpreting Statements with Multiple Quantifiers}

\slide{Meaning of Multiple Quantifiers}{
    Suppose \(P(x,y)\) denotes ``\(xy = yx\)'' (the commutative property). Assume that the domain is the set
    of real numbers. Then, the statement:
    \[\forall x \forall y \ P(x,y)\]
    means ``for all real numbers \(x\) and for all real numbers \(y\), \(xy = yx\).''
    Similarly,
    \[\forall y \forall x \ P(x,y)\]
    means ``for all real numbers \(y\) and for all real numbers \(x\), \(xy = yx\).''

    Fortunately, the order does \emph{not} matter in this case. However, the order
    will matter when two \emph{different} quantifiers are used in the same statement.
}


\slide{Meaning of Multiple Quantifiers}{
    Consider the true proposition ``All real numbers have an additive inverse.''
    \begin{enumerate}
        \item[(1)] What does \(\forall x \in \mathbb{R} \quad \exists y \in \mathbb{R} \quad (x+y = 0)\) mean?
        \item[(2)] What does \(\exists y \in \mathbb{R} \quad \forall x \in \mathbb{R} \quad (x+y = 0)\) mean?
    \end{enumerate}

    Here, (1) means ``for all real numbers \(x\), there exists a real number \(y\) such that,
    \(y\) is the additive inverse of \(x\).''

    Here, (2) means ``there exists a real number \(y\) such that, for all real numbers \(x\),
    \(y\) is the additive inverse of \(x\).''

    It is imperative to note that (1) and (2) do not mean the same thing.
}


\slide{Truth with Multiple Similar Quantifiers}{
    If you want to establish the truth of a statement of the form
    \[
    \forall x \in D \quad \forall y \in E \quad P(x,y)
    \]
    your challenge is to allow someone else to pick any element \(x \in D\) and
    any element \(y \in E\), and then you must show that \(P(x,y)\) holds for
    that arbitrary pair.

    If you want to establish the truth of a statement of the form
    \[
    \exists x \in D \quad \exists y \in E \quad P(x,y)
    \]
    your job is to find one particular \(x \in D\) and one particular
    \(y \in E\) for which \(P(x,y)\) is true.   
}


\slide{Truth with Multiple Different Quantifiers}{
    If you want to establish the truth of a statement of the form
    \[
    \forall x \in D \quad \exists y \in E \quad P(x,y)
    \]
    your challenge is to allow someone else to pick whatever element \(x \in D\) they wish
    and then you must find an element \(y \in E\) that ``works'' for that particular \(x\).
    You are allowed to find a different value of \(y\) for each different \(x\) you are given.

    If you want to establish the truth of a statement of the form
    \[
    \exists x \in D \text{ such that } \forall y \in E, \, P(x,y)
    \]
    your job is to find one particular \(x \in D\) that will ``work'' no matter what
    \(y \in E\) anyone might choose to challenge you with.
    You are not allowed to change your \(x\) once you have specified it initially.
}


\slide{Universal and Existential Quantifiers}{
    Consider the true proposition ``All real numbers have an additive inverse.''
    \begin{enumerate}
        \item[(1)] Is \(\forall x \in \mathbb{R} \quad \exists y \in \mathbb{R} \quad (x+y = 0)\) true?
        \item[(2)] Is \(\exists y \in \mathbb{R} \quad \forall x \in \mathbb{R} \quad (x+y = 0)\) true?
    \end{enumerate}
    
    We can say that (1) is true as it means that we are allowed to find a different value of \(y\) for
    each different \(x\) we are given.

    However, (2) is not true as it means that we have to find one value of \(y\) that satisfies
    \(x + y = 0\), as \(x\) changes.
}


\subsection{Negation of Statements with Multiple Quantifiers}

\slide{Negation of Statements with Multiple Quantifiers}{
    To find the negation of a statement with multiple quantifiers, we can use the same rules
    that we used to negate simpler quantified statements. Suppose the domain of \(x\) is \(D\),
    then recall that:
    \[\neg (\forall x \ P(x)) \equiv  \exists x \ \neg P(x)\]
    and
    \[\neg (\exists x \ P(x)) \equiv  \forall x \ \neg P(x).\]

    To find the negation of a statement with multiple quantifiers we apply the negation in steps
    like we do with an operator like \((-)\). For instance \[-(5 + 2 - 3) = -5 -(2-3) = -5 - 2 + 3.\]
}


\slide{Negation of Statements with Multiple Quantifiers}{
\[\begin{aligned}
    \neg \left[\forall x \in D \quad \forall y \in E \quad P(x,y)\right] &\equiv  \exists x \in D \quad \neg \left[\forall y \in E \quad P(x,y)\right] \\
    &\equiv  \exists x \in D \quad \exists y \in E \quad \neg P(x,y)
\end{aligned}\]

\[\begin{aligned}
    \neg \left[\forall x \in D \quad \exists y \in E \quad P(x,y)\right] &\equiv  \exists x \in D \quad \neg \left[\exists y \in E \quad P(x,y)\right] \\
    &\equiv  \exists x \in D \quad \forall y \in E \quad \neg P(x,y)
\end{aligned}\]

\[\begin{aligned}
    \neg \left[\exists x \in D \quad \forall y \in E \quad P(x,y)\right] &\equiv  \forall x \in D \quad \neg \left[\forall y \in E \quad P(x,y)\right] \\
    &\equiv  \forall x \in D \quad \exists y \in E \quad \neg P(x,y)
\end{aligned}\]

\[\begin{aligned}
    \neg \left[\exists x \in D \quad \exists y \in E \quad P(x,y)\right] &\equiv  \forall x \in D \quad \neg \left[\exists y \in E \quad P(x,y)\right] \\
    &\equiv  \forall x \in D \quad \forall y \in E \quad \neg P(x,y)
\end{aligned}\]
}


\subsection{Formal Logical Notation}

\slide{Formal Logical Notation}{
    The formal notation for logic involves using predicates to describe all properties of variables
    and omitting the words ``such that'' in existential statements.

    ``\(\forall x \text{ in } D\), \(P(x)\)'' can be written as ``\(\forall x \; (x \in D \to P(x))\).''

    ``\(\exists x \text{ in } D\) such that \(P(x)\)'' can be written as ``\(\exists x \; (x \in D \land P(x))\).''
}


\slide{Example}{
    Translate the statement “Every real number except zero has a multiplicative inverse.”
    (A multiplicative inverse of a real number \(x\) is a real number \(y\) such that \(xy = 1\).)

    \underline{Solution}:\\[0.5em]
    ``For every real \(x\), there exists \(y\), such that \(xy = 1.\)''
    \[\forall x \in \mathbb{R} - \{0\}, \exists y \in \mathbb{R} - \{0\} \text{ such that } xy = 1.\]
    Here, \(\exists y\) such that \(P(x,y)\) \(\equiv \exists y\left[y \in D \land P(x,y)\right]\).\\
    Furthermore, \(\forall x \ P(x,y)\) \(\equiv \forall x \left[x \in D \to P(x,y)\right]\).
    \[\forall x \left[(x \in \RN \land x \not= 0) \to \exists y (xy = 1)\right].\]
}