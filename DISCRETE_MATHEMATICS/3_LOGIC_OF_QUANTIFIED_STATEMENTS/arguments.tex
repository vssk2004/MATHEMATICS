\section{Arguments with Quantified Statements}


\subsection{Rules of Inference for Quantified Statements}

\slide{Rules of Inference for Quantified Statements}{
    We have discussed rules of inference for propositions. We will now describe some important
    rules of inference for statements involving quantifiers. These rules of inference are used extensively in mathematical arguments.

    Here, we will discuss four rules:
    \begin{enumerate}
        \item[(1)] Universal Instantiation
        \item[(2)] Universal Generalization
        \item[(3)] Existential Instantiation
        \item[(4)] Existential Generalization
    \end{enumerate}
}


\slide{Universal Instantiation}{
    \dfn{\textbf{Universal instantiation} is the rule of inference used to conclude that \(P(c)\) is true, where \(c\)
    is a particular member of the domain, given the premise \(\forall x P(x)\).}

    Universal instantiation means that the truth of a property in a particular case
    follows as a special instance of its more general or universal truth.
    Consider one of the most famous  examples of universal instantiation:
    \[\begin{aligned}
    &\text{All men are mortal.} \\
    &\text{Socrates is a man.} \\
    \therefore \ &\text{Socrates is mortal.}
    \end{aligned}\]
}


\slide{Universal Generalization}{
    \dfn{\textbf{Universal generalization} is the rule of inference that states that \(\forall x P(x)\) is true, given the
    premise that \(P(c)\) is true for all elements \(c\) in the domain.}

    If you prove something without assuming anything special about the element, you may conclude it holds universally.

    The element \(c\) that we select must be an arbitrary, and not a specific, element of the domain. We have no control over \(c\) and cannot
    make any other assumptions about \(c\) other than it comes from the domain.

    For example, we can prove that \(x + 0 = x\) for any real number. To do so, we make no specific assumptions about \(x\).
}


\slide{Existential Instantiation}{
    \dfn{\textbf{Existential instantiation} is the rule that allows us to conclude that there is an element \(c\) in
    the domain for which \(P(c)\) is true if we know that \(\exists xP(x)\) is true.}

    Usually we have no knowledge of what \(c\) is, only that it exists. However, because it exists, we may give it a name (\(c\))
    and continue our argument.
}


\slide{Existential Generalization}{
    \dfn{\textbf{Existential generalization} is the rule of inference that is used to conclude that \(\exists xP(x)\) is
    true when a particular element \(c\) with \(P(c)\) true is known.}

    If we know one element \(c\) in the domain for which \(P(c)\) is true, then we know that \(\exists xP(x)\) is true.
}


\slide{Summary of Rules of Inference for Quantified Statements}{
\begin{center}
\renewcommand{\arraystretch}{1.5}
\begin{tabular}{|c|c|}
    \hline
    \textbf{Universal Instantiation} & \textbf{Universal Generalization} \\
    \hline
    \(\forall x P(x)\) & \(P(c)\) for all \(c\) in the domain \\
    \(\therefore P(c)\) & \(\therefore \forall x P(x)\) \\
    \hline
    \textbf{Existential Instantiation} & \textbf{Existential Generalization} \\
    \hline
    \(\exists x P(x)\) & \(P(c)\) for some \(c\) in the domain \\
    \(\therefore P(c)\) & \(\therefore \exists x P(x)\) \\
    \hline
\end{tabular}
\end{center}
}


\slide{Example 1}{
    Consider the following exercise:
    \begin{center}
        Show that the premises “Everyone in this discrete mathematics class has taken a course in
        computer science” and “Marla is a student in this class” imply the conclusion “Marla has taken
        a course in computer science.”
    \end{center}

    Suppose \(D(x)\) is ``\(x\) is in this discrete mathematics class,'' and let \(C(x)\) denote ``\(x\) has
    taken a course in computer science.'' Then the premises are \(\forall x (D(x) \to C(x))\) and \(D(Marla)\).
    The conclusion is \(C(Marla)\).

    How do we get from the premises to the conclusion? We use rules of inference for quantified statements
    and modus ponens.
}


\slide{Example 1}{
\begin{tabular}{@{}p{0.4\linewidth} p{0.5\linewidth}@{}}
\textbf{Step} & \textbf{Reason} \\[6pt]

1.\ $\forall x (D(x) \rightarrow C(x))$ 
& Premise \\[4pt]

2.\ $D(\text{Marla}) \rightarrow C(\text{Marla})$ 
& Universal Instantiation from (1) \\[4pt]

3.\ $D(\text{Marla})$ 
& Premise \\[4pt]

4.\ $C(\text{Marla})$ 
& Modus Ponens from (2) and (3)
\end{tabular}
}


\subsection{Combining Rules of Inference for Propositions and Quantified Statements}

\slide{Universal Modus Ponens}{
    Note that in our argument in Example 1 we used both universal instantiation, a rule of inference
    for quantified statements, and modus ponens, a rule of inference for propositional logic.

    We will often need to use this combination of rules of inference. Because universal instantiation
    and modus ponens are used so often together, this combination of rules is sometimes called
    \textbf{universal modus ponens}.

    \[\begin{aligned}
        &\forall x (P(x) \to Q(x)) \\
        &P(c) \\
        \therefore \ &Q(c)
    \end{aligned}\]
}


\slide{Example}{
    Assume that “For all positive integers \(n\), if \(n\) is greater than 4, then \(n^2\) is less than \(2^n\)” is true.
    Use universal modus ponens to show that \(100^2 < 2^{100}\).

    \underline{Solution}:\\[0.5em]
    Suppose that \(P(x)\) is ``\(x > 4\),'' and \(Q(x)\) is ``\(x^2 < 2^x\).'' Furthermore, let \(c = 100\).
    Then by universal modus ponens:
    \[\begin{aligned}
        &\forall x (P(x) \to Q(x)) \\
        &P(100) \\
        \therefore \ &Q(100).
    \end{aligned}\]
}


\slide{Universal Modus Tollens}{
    Another useful combination of a rule of inference from propositional logic and a rule of inference for quantified
    statements is \textbf{universal modus tollens}.

    Universal modus tollens combines universal instantiation and modus tollens and can be expressed in the following way:
    \[\begin{aligned}
        &\forall x (P(x) \to Q(x)) \\
        &\neg Q(c) \\
        \therefore \ &\neg P(c)
    \end{aligned}\]
}


\slide{Example}{
    \[\begin{aligned}
        &\text{All human beings are mortal.} \\
        &\text{Zeus is not mortal.} \\
        \therefore \ &\text{Zeus is not a human being.}
    \end{aligned}\]
}