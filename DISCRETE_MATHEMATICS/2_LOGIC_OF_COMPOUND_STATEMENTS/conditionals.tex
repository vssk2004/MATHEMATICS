\section{Conditional Propositions}


\subsection{Conditional}

\slide{Conditional}{
    \dfn{If \(p\) and \(q\) are propositional variables, the the conditional of \(q\) by \(p\) is “If \(p\), then \(q\).”
    It is denoted by \(p \to q\).
    It is false when \(p\) is true and \(q\) is false; otherwise it is true.
    We call \(p\) the hypothesis (or antecedent) of the conditional and \(q\) the conclusion (or consequent).}

    The notation \(p \to q\) indicates that \(\to\) is a connective, like \(\land\) or \(\lor\), which can be used to join propositions to create new propositions.

    To define \(p \to q\) as a statement, therefore, we must specify the truth values for \(p \to q\) as we specified truth values for \(p \land q\) and for \(p \lor q\).
}


\slide{Truth Table for Conditional}{
    To understand the truth table for \(p \to q\), we can think of the conditional as a promise.
    For instance, suppose you go to interview for a job at a store and the owner of the
    store makes you the promise: ``If you show up for work Monday morning, then you will get the job.''

    Under what circumstances are you justified in saying the owner spoke
    falsely? That is, under what circumstances is the above sentence false?
    The answer is: You \emph{do} show up for work Monday morning and you \emph{do
    not} get the job. Nothing is specified about what happens if you do not show
    up. Therefore, it is not false regardless of whether you get the job or not.

    \begin{center}
    \resizebox{0.225\textwidth}{!}{%
    \begin{tabular}{|c|c|c|}
    \hline
    $p$ & $q$ & $p \to q$ \\
    \hline
    T & T & T \\
    T & F & F \\
    F & T & T \\
    F & F & T \\
    \hline
    \end{tabular}%
    }
    \end{center}
}


\slide{Vacuously True}{
    A conditional statement that is true by virtue of the fact that its hypothesis is false is often
    called \textbf{vacuously true} or \textbf{true by default}.

    In the previous example, the store owner's promise only says you will get the job if a certain
    condition (showing up for work Monday morning) is met; it says nothing about what will happen if
    the condition is not met.

    For example, consider the proposition ``If \(0 = 1\), then \(1 = 2\).'' This proposition is 
    true. Why? The hypothesis \(0 = 1\) is false, therefore the conditional is vacuously true.
}


\slide{Updated Order of Operations}{
    The order is as follows:
    \begin{enumerate}
        \item[(1)] \(\neg\) 
        \item[(2)] \(\land\) and \(\lor\)
        \item[(3)] \(\to\) 
    \end{enumerate}
}


\slide{Representation of If-Then as Or}{
    Consider the conditional \(p \to q\). We can rewrite this as \(\neg p \lor q\).

    \begin{center}
    \resizebox{0.5\textwidth}{!}{%
    \begin{tabular}{|c c|c|c c|}
    \hline
    $p$ & $q$ & $\neg p$ & $\neg p \lor q$ & $p \to q$ \\
    \hline
    T & T & F & T & T \\
    T & F & F & F & F \\
    F & T & T & T & T \\
    F & F & T & T & T \\
    \hline
    \end{tabular}%
    }
    \end{center}

    As evident, \(\neg p \lor q\) and \(p \to q\) have the same truth values for all
    possible combinations of truth values of \(p\) and \(q\). Therefore, they are logically equivalent.
}


\subsection{Negation of Conditional}

\slide{Negation of a Conditional}{
    We now know that \(p \to q \; \equiv \; \neg p \lor q\). Thus, we can now find the negation of \(p \to q\) by
    taking the negation of \(\neg p \lor q\).

    \[
    \neg (\neg p \lor q) \; \ \equiv \; \ \neg (\neg p) \land \neg q \; \ \equiv \; \ p \land \neg q
    \]
    \[\therefore \; \; \neg(p \to q) \; \equiv \; p \land \neg q\]

    The negation of “if \(p\) then \(q\)” is logically equivalent to “\(p\) and not \(q\).”
}


\slide{Example}{
    Write down the negations of the following conditionals:
    \begin{enumerate}
        \item[(1)] If my car is in the repair shop, then I cannot get to class.
        \item[(2)] If it is sunny, then I will go to the beach. 
    \end{enumerate}

    Answers:
    \begin{enumerate}
        \item[(1)] My car is in the repair shop and I can get to class.
        \item[(2)] It is sunny and I will not go to the beach.
    \end{enumerate}
}


\subsection{Contrapositive}

\slide{Contrapositive}{
    \dfn{The \textbf{contrapositive} of a conditional proposition of form ``If \(p\) then \(q\)'' is:
    \[\text{If} \ \neg q \ \text{then} \ \neg p.\]
    Symbolically, the contrapositive of \(p \to q\) is \(\neg q \to \neg p\).}

    A conditional proposition is logically equivalent to its contrapositive.

    The above stated result is a fundamental result in logic. It is used very often in mathematical proofs
    as a proof technique called \textbf{proof by contrapositive}, which will be discussed later.
}


\subsection{Converse and Inverse}

\slide{Converse}{
    \dfn{Suppose a conditional proposition of the form “If p then q” is given.
    Its \textbf{converse} is:
    \[\text{If} \ q \ \text{then} \ p.\]
    Symbolically, the converse of \(p \to q\) is \(q \to p\).}

    \dfn{Suppose a conditional proposition of the form “If p then q” is given.
    Its \textbf{inverse} is:
    \[\text{If} \ \neg p \ \text{then} \ \neg q.\]
    Symbolically, the inverse of \(p \to q\) is \(\neg p \to \neg q\).}
}


\slide{Example}{
    Write the converse and inverse of each of the following propositions:
    \begin{enumerate}
        \item[(1)] If Howard can swim across the lake, then Howard can swim to the island.
        \item[(2)] If today is Easter, then tomorrow is Monday.
    \end{enumerate}

    Answers:
    \begin{itemize}
        \item Converse (1): If Howard can swim to the island, then Howard can swim across the lake.
        \item Inverse (1): If Howard cannot swim across the lake, then Howard cannot swim to the island.
        \item Converse (2): If tomorrow is Monday, then today is Easter.
        \item Inverse (2): If today is not Easter, then tomorrow is not Monday.
    \end{itemize}
}


\slide{Converse and Inverse}{
    Note that the converse and inverse of a conditional proposition are logically equivalent.
    For this, suppose that we have a conditional proposition \(p \to q\).

    Its converse is: \(q \to p\). The contrapositive of the converse is: \(\neg p \to \neg q\).
    Note the contrapositive of the converse is the inverse.

    Since a conditional proposition and its contrapositive are logically equivalent, the
    converse and the inverse of a conditional are logcially equivalent.
}


\subsection{Biconditional}

\slide{Only If}{
    It is important to understand ``only if '' before studying the biconditional.
    
    To say “\(p\) only if \(q\)” means that \(p\) can take place only if \(q\) takes place also.
    That is, if \(q\) does not take place, then \(p\) cannot take place.

    Symbolically, “\(p\) only if \(q\)” means \(\neg q \to \neg p\). Furthermore, we know
    that a conditional proposition is logically equivalent to its contrapositive.
    
    Therefore, “\(p\) only if \(q\)” means \(p \to q\).
}


\slide{Caution!}{
    Remember, “\(p\) only if \(q\)” does \emph{not} mean ``\(p\) if \(q\).''

    \[p \text{ only if } q \; \text{ means } \; p \to q\]
    \[p \text{ if } q \; \text{ means } \; q \to p\]
}


\slide{Biconditional}{
    \dfn{Given propositional variables \(p\) and \(q\), the \textbf{biconditional} of \(p\) and
    \(q\) is ``\(p\) if, and only if, \(q\).'' It is denoted by \(p \leftrightarrow q\).
    It is true if both \(p\) and \(q\) have the same truth values and is false if \(p\) and \(q\)
    have opposite truth values.}

    Note that the first part of the biconditional is ``\(p\) if \(q\)'' this means \(q \to p\).
    The second part is ``\(p\) only if \(q\)'' this means \(p \to q\).

    The phrase ``if and only if'' is abbreviated \textbf{iff}.

    \begin{center}
    \resizebox{0.225\textwidth}{!}{%
    \begin{tabular}{|c c|c|}
    \hline
    $p$ & $q$ & $p \leftrightarrow q$ \\
    \hline
    T & T & T \\
    T & F & F \\
    F & T & F \\
    F & F & T \\
    \hline
    \end{tabular}%
    }
    \end{center}
}


\slide{Order of Operations}{
    The order is as follows:
    \begin{enumerate}
        \item[(1)] \(\neg\) 
        \item[(2)] \(\land\) and \(\lor\)
        \item[(3)] \(\to\)  and \(\leftrightarrow\)
    \end{enumerate}
}


\slide{Expressing Biconditional using Conditionals}{
    Note that the first part of the biconditional is ``\(p\) if \(q\)'' this means \(q \to p\).
    The second part is ``\(p\) only if \(q\)'' this means \(p \to q\). Therefore, the biconditional
    is simply the conjunction of the two conditionals.

    \begin{center}
    \resizebox{0.7\textwidth}{!}{%
    \begin{tabular}{|c c|c|c|c|c|}
    \hline
    $p$ & $q$ & $p \to q$ & $ q\to p$ & $p \leftrightarrow q$ &  \((p \to q) \land (q \to p)\) \\
    \hline
    T & T & T & T & T & T\\
    T & F & F & T & F & F\\
    F & T & T & F & F & F\\
    F & F & T & T & T & T\\
    \hline
    \end{tabular}%
    }
    \end{center}
}


\subsection{Necessary and Sufficient Conditions}

\slide{Necessary and Sufficient Conditions}{
    \dfn{If \(p\) and \(q\) are propositional variables:
    \begin{itemize}
        \item ``\(p\) is a \textbf{sufficient condition} for \(q\)'' means that \(p \to q\)
        \item ``\(p\) is a \textbf{necessary condition} for \(q\)'' means that \(q \to p\) 
    \end{itemize}}

    The sufficient condition is equivalent to ``\(p\) only if \(q\)'' and the necessary condition is equivalent
    to ``\(p\) if \(q\).''

    Therefore, saying that ``\(p\) is a necessary and sufficient condition for \(q\)'' means
    ``\(p\) if and only if \(q\).'' Symbolically, it means \(p \leftrightarrow q\).
}