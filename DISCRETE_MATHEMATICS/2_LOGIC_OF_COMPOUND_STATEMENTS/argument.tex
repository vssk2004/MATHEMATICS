\section{Valid and Invalid Arguments}


\subsection{Arguments}

\slide{Arguments and Argument Form}{
    \dfn{An \textbf{argument} in logic is a sequence of propositions. All but the final proposition
    in the argument are called \textbf{premises} and the final proposition is called the \textbf{conclusion}.
    An argument is \textbf{valid} if the truth of all its premises implies that the conclusion is true.}

    \dfn{An \textbf{argument form} in propositional logic is a sequence of compound propositions involving
    propositional variables. An argument form is \textbf{valid} if no matter which particular
    propositions are substituted for the propositional variables in its premises, the conclusion is
    true if the premises are all true.}
}


\subsection{Validity}

\slide{Valid Argument}{
    The crucial fact about a valid argument is that the truth of its conclusion follows \emph{necessarily}
    or \emph{inescapably} or \emph{by logical form alone} from the truth of its premises.

    It is impossible to have a valid argument with all true premises and a false conclusion.

    When an argument is valid and its premises are true, the truth of the conclusion is said to be \emph{inferred}
    or \emph{deduced} from the truth of its premises.
}


\slide{Testing Validity}{
    We now understand that when an argument is valid, the truth of the conclusion is inferred from the truth of its
    premises. However, we might sometimes want to confirm whether an argument is valid. To achieve this, we use
    truth tables.

    To test an argument form for validity:
    \begin{enumerate}
        \item[(1)] Identify the premises and conclusion of the argument form.
        \item[(2)] Construct a truth table showing the truth values of all the premises and the conclusion.
        \item[(3)] A row of the truth table in which all the premises are true is called a \emph{critical row}.
        If the conclusion in every critical row is true, then the argument form is valid. Otherwise, it is invalid.
    \end{enumerate}
}


\slide{Example}{
    Prove that the following argument form is valid:
    \[\begin{aligned}
        &\text{If } p \text{ then } q. \\
        &p. \\
        \therefore \ &q.
    \end{aligned}\]

    \begin{center}
    \resizebox{0.4\textwidth}{!}{%
    \begin{tabular}{|c|c|c|c||c|}
    \cline{3-5}
    \multicolumn{2}{c|}{} 
    & \multicolumn{2}{c||}{\textbf{premises}} 
    & \multicolumn{1}{c|}{\textbf{conclusion}} \\
    \hline
    $p$ & $q$ & $p \rightarrow q$ & $p$ & $q$ \\
    \hline
    \rowcolor{blue!15}
    T & T & T & T & T \\
    \hline
    T & F & F & T & F \\
    \hline
    F & T & T & F & T \\
    \hline
    F & F & T & F & F \\
    \hline
    \end{tabular}
    }
    \end{center}

    As evident from the highlighted critical row, the conclusion in every critical row is true, therefore the
    argument form is valid.
}


\slide{Validity and Truth Tables}{
    We can always use a truth table to show that an argument form is valid. We do this by showing
    that whenever the premises are true, the conclusion must also be true. However, this can be
    a tedious approach.
    
    For example, when an argument form involves 10 different propositional variables, to use a truth
    table to show this argument form is valid requires \(2^{10} = 1024\) different rows.

    Fortunately, we do not always have to resort to truth tables. Instead, we can first establish the
    validity of some relatively simple argument forms, called \textbf{rules of inference}.

    We will now introduce \emph{modus ponens} and \emph{modus tollens}, the most important rules of inference in
    propositional logic.
}


\subsection{Modus Ponens and Modus Tollens}

\slide{Syllogism}{
    \dfn{An argument form consisting of two premises and a conclusion is called a \textbf{syllogism}.
    The first and second premises are called the major premise and minor premise, respectively.}

    The two most famous forms of syllogism in logic are modus ponens and modus tollens.
}


\slide{Modus Ponens}{
    Modus Ponens has the form:
    \[\boxed{\begin{aligned}
        &p \to q. \\
        &p. \\
        \therefore \ &q.
    \end{aligned}}\]
    
    Here is an argument of this form:
    \[
    \begin{aligned}
    &\text{If the sum of the digits of 371,487 is divisible by 3,} \\
    &\text{then 371,487 is divisible by 3.} \\
    &\text{The sum of the digits of 371,487 is divisible by 3.} \\
    \therefore\ &\text{371,487 is divisible by 3.}
    \end{aligned}
    \]

    The term modus ponens is Latin meaning ``method of affirming'' (the conclusion is an affirmation).
}


\slide{Proving Validity of Modus Ponens}{
    \begin{center}
    \resizebox{0.5\textwidth}{!}{%
    \begin{tabular}{|c|c|c|c||c|}
    \cline{3-5}
    \multicolumn{2}{c|}{} 
    & \multicolumn{2}{c||}{\textbf{premises}} 
    & \multicolumn{1}{c|}{\textbf{conclusion}} \\
    \hline
    $p$ & $q$ & $p \rightarrow q$ & $p$ & $q$ \\
    \hline
    \rowcolor{blue!15}
    T & T & T & T & T \\
    \hline
    T & F & F & T & F \\
    \hline
    F & T & T & F & T \\
    \hline
    F & F & T & F & F \\
    \hline
    \end{tabular}
    }
    \end{center}

    The first row is the only one in which both premises are true. Furthermore, the conclusion in that 
    row is also true. Hence, the argument form is valid.
}


\slide{Modus Tollens}{
    Modus Tollens has the form:
    \[\boxed{\begin{aligned}
        &p \to q. \\
        &\neg q. \\
        \therefore \ & \neg p.
    \end{aligned}}\]

    Here is an argument of this form:
    \[
    \begin{aligned}
    &\text{If Zeus is human, then Zeus is mortal.} \\
    &\text{Zeus is not mortal.} \\
    \therefore\ &\text{Zeus is not human.}
    \end{aligned}
    \]

    The term modus tollens is Latin meaning ``method of denying'' (the conclusion is a denial).
}


\slide{Proving Validity of Modus Tollens}{
    \begin{center}
    \resizebox{0.5\textwidth}{!}{%
    \begin{tabular}{|c|c|c|c||c|}
    \cline{3-5}
    \multicolumn{2}{c|}{} 
    & \multicolumn{2}{c||}{\textbf{premises}} 
    & \multicolumn{1}{c|}{\textbf{conclusion}} \\
    \hline
    $p$ & $q$ & $p \rightarrow q$ & $\neg q$ & $\neg p$ \\
    \hline
    T & T & T & F & F \\
    \hline
    T & F & F & T & F \\
    \hline
    F & T & T & F & T \\
    \hline
    \rowcolor{blue!15}
    F & F & T & T & T \\
    \hline
    \end{tabular}
    }
    \end{center}

    The last row is the only one in which both premises are true. Furthermore, the conclusion in that 
    row is also true. Hence, the argument form is valid.
}


\subsection{Rules of Inference}

\slide{Validity of an Argument}{
    As we have seen, the rules of inferences called modus ponens and modus tollens are valid forms of argument. Now, we look
    at more rules of inference. 
    
    Before that, it is imperative to understand that
    from the definition of a valid argument form, we see that the argument form with premises \(p_1, p_2, \dots, p_n\)
    and conclusion \(q\) is valid exactly when \(p_1 \land p_2 \land \dots \land p_n \to q\) is a tautology.
}


\slide{Rules of Inference}{
\begin{table}[h!]
\centering
\resizebox{0.65\textwidth}{!}{%
\begin{tabular}{|>{\centering\arraybackslash}m{4cm}|
                >{\centering\arraybackslash}m{5.5cm}|
                >{\centering\arraybackslash}m{4cm}|}
\hline
\textbf{Rule of Inference} & \textbf{Tautology} & \textbf{Name} \\ \hline

$\begin{aligned}
p \to q \\
p \\
\therefore\ q
\end{aligned}$
&
$(p \land (p \to q)) \to q$
&
Modus ponens \\ \hline

$\begin{aligned}
p \to q \\
\neg q \\
\therefore\ \neg p
\end{aligned}$
&
$(\neg q \land (p \to q)) \to \neg p$
&
Modus tollens \\ \hline

$\begin{aligned}
p \to q \\
q \to r \\
\therefore\ p \to r
\end{aligned}$
&
$((p \to q) \land (q \to r)) \to (p \to r)$
&
Hypothetical syllogism (Transitivity) \\ \hline

$\begin{aligned}
p \lor q \\
\neg p \\
\therefore\ q
\end{aligned}$
&
$((p \lor q) \land \neg p) \to q$
&
Disjunctive syllogism (Elimination) \\ \hline

$\begin{aligned}
p \\
\therefore\ p \lor q
\end{aligned}$
&
$p \to (p \lor q)$
&
Addition (Generalization) \\ \hline

$\begin{aligned}
p \land q \\
\therefore\ p
\end{aligned}$
&
$(p \land q) \to p$
&
Simplification (Specialization) \\ \hline

$\begin{aligned}
p \\
q \\
\therefore\ p \land q
\end{aligned}$
&
$(p \land q) \to (p \land q)$
&
Conjunction \\ \hline

$\begin{aligned}
p \lor q \\
\neg p \lor r \\
\therefore\ q \lor r
\end{aligned}$
&
$((p \lor q) \land (\neg p \lor r)) \to (q \lor r)$
&
Resolution \\ \hline

\end{tabular}%
}
\caption{Rules of Inference}
\end{table}
}


\subsection{Fallacy}

\slide{Fallacy}{
    A \textbf{fallacy} is an error in reasoning that results in an invalid argument.
    
    Some common fallacies include:
    \begin{itemize}
        \item using ambiguous premises (and treating them as if they were unambiguous)
        \item circular reasoning (assuming what is to be proved without having derived it from the premises)
        \item jumping to a conclusion (without adequate grounds)
        \item converse error (i.e. fallacy of affirming the conclusion)
        \item inverse error (i.e. fallacy of denying the hypothesis)
    \end{itemize}
}


\slide{Converse Error}{
    Consider the following argument:
    \[\begin{aligned}
        &\text{If you do every problem in the book, then you will learn discrete} \\
        &\text{mathematics.} \\
        &\text{You learned discrete mathematics.} \\
        &\text{Therefore, you did every problem in the book.}
    \end{aligned}\]

    Is this argument valid? The answer is \textbf{no}. Why?

    The argument's premises are \(p \to q\) and \(q\). Its conclusion is \(p\). This argument form is
    valid when \(((p \to q) \land q) \to p\) is a tautology. However, we can prove that this
    is \emph{not} a tautology using truth tables.

    This type of incorrect reasoning is called the fallacy of affirming the conclusion or converse error.
}


\slide{Inverse Error}{
    Consider the following argument:
    \[\begin{aligned}
        &\text{If you do every problem in the book, then you will learn discrete} \\
        &\text{mathematics.} \\
        &\text{You did not do every problem in the book.} \\
        &\text{Therefore, you did not learn discrete mathematics.}
    \end{aligned}\]

    Is this argument valid? The answer is \textbf{no}. Why?

    The argument's premises are \(p \to q\) and \(\neg p\). Its conclusion is \(\neg q\). This argument form is
    valid when \(((p \to q) \land \neg p) \to \neg q\) is a tautology. However, we can prove that this
    is \emph{not} a tautology using truth tables.

    This type of incorrect reasoning is called the fallacy of denying the hypothesis or inverse error.
}

\subsection{Contradictions and Valid Arguments}

\slide{Contradiction}{
    Contradiction can be used to make inferences through a technique of reasoning called the \emph{contradiction rule}.

    \dfn{If it can be shown that the proposition \(\neg p\) leads to a contradiction, then we can conclude
    that \(p\) is true. Symbolically, we must show that \(\neg p \to c\).}

    Now, it is natural to ask whether or not this is a valid argument form. It is.
    This can be shown using the truth table for the following argument form which
    represents the contradiction rule:
    \[\begin{aligned}
        & \neg p \to c \\
        \therefore \ &p
    \end{aligned}\]
}


\slide{Contradiction Rule}{
    Recall that an argument form with premises \(p_1, p_2, \dots, p_n\) and conclusion \(q\) is valid when
    \((p_1 \land p_2 \land \dots \land p_n) \to q\) is a tautology.

    Here, the premise is \(\neg p \to c\) and the conclusion is \(p\). Hence, we must show that \((\neg p \to c) \to p\)
    is a tautology to prove that the contradiction rule is valid.

    \begin{center}
    \resizebox{0.5\textwidth}{!}{%
    \begin{tabular}{|c|c|c|c|c|}
    \hline
    $p$ & $\neg p$ & $c$ & $\neg p \to c$ & $(\neg p \to c) \to p$ \\
    \hline
    T & F & F & T & T \\
    \hline
    F & T & F & F & T \\
    \hline
    \end{tabular}
    }
    \end{center}

    As evident, \((\neg p \to c) \to p\) is a tautology. Therefore, the argument form of the contradiction rule
    is valid.
    
    The contradiction rule is the logical heart of the method of proof by contradiction.
}


\section{Introduction to Boolean Expressions}

\slide{Boolean Variables and Boolean Expressions}{
    In logic, variables such as \(p\), \(q\), and \(r\) represent propositions. Furthermore, a proposition
    can have one of only two truth values: T (true) or F (false).

    A propositional form (or compound proposition) is an expression, such as
    \(p \land (\neg q \lor r)\), composed of propositional variables and logical connectives.

    One of the founders of symbolic logic was the English mathematician George Boole. In his honor,
    any variable, such as a propositional variable or an input signal, that can take one of
    only two values is called a \textbf{Boolean variable}.

    An expression composed of Boolean variables and the connectives \(\neg\) , \(\land\), and \(lor\) is
    called a \textbf{Boolean expression}.
}