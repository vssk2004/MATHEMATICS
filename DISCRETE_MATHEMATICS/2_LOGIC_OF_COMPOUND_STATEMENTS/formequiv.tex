\section{Logical Form and Logical Equivalence}


\subsection{Overview}

\slide{Arguments, Form, and Conclusions}{
    An \textbf{argument} is a sequence of statements aimed at proving the truth of an assertion.
    The assertion at the end of the sequence is called the \textbf{conclusion}, and the preceding
    statements are called \textbf{premises}.

    In logic, the form of an argument is distinguished from its content. This means that logical analysis
    won't help you determine the truth of an argument's content, but it will help
    you analyze an argument's form to determine whether the truth of the conclusion follows
    necessarily from the truth of the premises.
}


\slide{Arguments, Form, and Conclusions}{
    Consider the following two arguments.
    \begin{center}
        Argument 1
    \end{center}
    {\scriptsize
    \[\text{If } \underbrace{\text{the bell rings}}_{p}
    \ \text{or} \
    \underbrace{\text{the flag drops}}_{q},
    \ \text{then} \
    \underbrace{\text{the race is over}}_{r}.\]}
    {\scriptsize
    \[
    \therefore \ 
    \text{If }\underbrace{\text{the race is not over}}_{\text{not } r},
    \ \text{then} \
    \underbrace{\text{the bell hasn’t rung}}_{\text{not } p}
    \ \text{and} \
    \underbrace{\text{the flag hasn’t dropped}}_{\text{not } q}.\]}

    \begin{center}
        Argument 2
    \end{center}
    {\scriptsize
    \[\text{If } \underbrace{ x=2}_{p}
    \ \text{or} \
    \underbrace{x=-2}_{q},
    \ \text{then} \
    \underbrace{x^2 = 4}_{r}.\]}
    {\scriptsize
    \[\therefore
    \text{If } \underbrace{ x^2 \not= 4,}_{\text{not } r}
    \ \text{then} \
    \underbrace{x \not= 2}_{\text{not } p}
    \ \text{and} \
    \underbrace{x \not= -2}_{\text{not } q}.\]}
}


\slide{Arguments, Form, and Conclusions}{
    They have very different content but their logical form is the same. To help make
    this clear, we use letters like \(p\), \(q\), and \(r\) to represent component sentences.
    We let the expression “not \(p\)” refer to the sentence ``It is not the case that \(p\) is true''
    or ``\(p\) is false''.

    The two arguments have a common \emph{logical form} of:
    \begin{center}
        If \(p\) or \(q\), then \(r\). \\
        \(\therefore\) If not \(r\), then not \(p\) and not \(q\).
    \end{center}
}


\subsection{Propositions}

\slide{Proposition}{
    \dfn{A \textbf{proposition} is a sentence that is either true or false but not both.}
    \begin{example}
        For example, ``2 + 2 = 4'' and ``2 + 2 = 5'' are both propositions,
        the first because it is true and the second because it is false.
    \end{example}

    \dfn{A \textbf{theorem} is a proposition that can be shown to be true.}
    \begin{example}
        For example, Pythagoras' theorem ``\(a^2 + b^2 = c^2\)'' can be shown to be true
        in a multitude of ways.
    \end{example}
}


\slide{Propositions}{
    Here are some examples of sentences that are \emph{not} propositions:
    \begin{enumerate}
        \item[(1)] \(x^2 + 2 = 11\)
        \item[(2)] \(x + y > 0\)
    \end{enumerate}

    (1) is not a proposition because its truth or falsity depends
    on the value of \(x\). For some values of \(x\), it is true, whereas for other values it is false.

    (2) is not a proposition because its truth or falsity depends
    on the values of \(x\) and \(y\). For instance, when \(x = -1\) and \(y = 2\) it is true, whereas
    when \(x = -1\) and \(y = 1\) it is false.
}


\subsection{Compound Propositions}

\slide{Compound Propositions}{
    We have seen that propositions are sentences that are either true or false but not both.

    Naturally, we would next like to build more complex sentences.
    We now introduce three symbols that are used to build more complicated logical expressions\
    out of simpler ones.
}

\slide{Negation}{
    The symbol \(\neg\) denotes ``not''. Given a proposition \(p\), the sentence \(\neg p\) is read
    ``not \(p\)'' or “it is not the case that \(p\) is true'' or ``\(p\) is false.''

    More formally, it called the \textbf{negation} of \(p\). It can be written in many different ways
    including: \(\sim p\), \(\bar{p}\), \(!p\), or even \(p'\).
}


\slide{Disjunction}{
    The symbol \(\lor\) denotes ``or''. Given propositions \(p\) and \(q\), the sentence \(p \lor q\)
    is read ``\(p\) or \(q\).''

    More formally, \(p \lor q\) is called the \textbf{disjunction} of \(p\) and \(q\). This can also be written
    as: \(p + q\), \(p \ | \ q\), and \(p \ || \ q\).

    Programming languages often use \(p \ || \ q\).
}


\slide{Conjunction}{
    The symbol \(\land\) denotes ``and''. Given propositions \(p\) and \(q\), the sentence \(p \land q\)
    is read ``\(p\) and \(q\).''

    More formally, \(p \land q\) is called the \textbf{conjunction} of \(p\) and \(q\). This can also be written
    as: \(p \times q\), \(p \ \& \ q\), \(p \ \& \& \ q\).

    Programming languages often use \(p \ \& \& \ q\).
}