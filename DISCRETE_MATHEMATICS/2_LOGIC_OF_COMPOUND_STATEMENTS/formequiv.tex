\section{Logical Form and Logical Equivalence}


\subsection{Overview}

\slide{Arguments, Form, and Conclusions}{
    An \textbf{argument} is a sequence of statements aimed at proving the truth of an assertion.
    The assertion at the end of the sequence is called the \textbf{conclusion}, and the preceding
    statements are called \textbf{premises}.

    In logic, the form of an argument is distinguished from its content. This means that logical analysis
    won't help you determine the truth of an argument's content, but it will help
    you analyze an argument's form to determine whether the truth of the conclusion follows
    necessarily from the truth of the premises.
}


\slide{Arguments, Form, and Conclusions}{
    Consider the following two arguments.
    \begin{center}
        Argument 1
    \end{center}
    {\scriptsize
    \[\text{If } \underbrace{\text{the bell rings}}_{p}
    \ \text{or} \
    \underbrace{\text{the flag drops}}_{q},
    \ \text{then} \
    \underbrace{\text{the race is over}}_{r}.\]}
    {\scriptsize
    \[
    \therefore \ 
    \text{If }\underbrace{\text{the race is not over}}_{\text{not } r},
    \ \text{then} \
    \underbrace{\text{the bell hasn’t rung}}_{\text{not } p}
    \ \text{and} \
    \underbrace{\text{the flag hasn’t dropped}}_{\text{not } q}.\]}

    \begin{center}
        Argument 2
    \end{center}
    {\scriptsize
    \[\text{If } \underbrace{ x=2}_{p}
    \ \text{or} \
    \underbrace{x=-2}_{q},
    \ \text{then} \
    \underbrace{x^2 = 4}_{r}.\]}
    {\scriptsize
    \[\therefore
    \text{If } \underbrace{ x^2 \not= 4,}_{\text{not } r}
    \ \text{then} \
    \underbrace{x \not= 2}_{\text{not } p}
    \ \text{and} \
    \underbrace{x \not= -2}_{\text{not } q}.\]}
}


\slide{Arguments, Form, and Conclusions}{
    They have very different content but their logical form is the same. To help make
    this clear, we use letters like \(p\), \(q\), and \(r\) to represent component sentences.
    We let the expression “not \(p\)” refer to the sentence ``It is not the case that \(p\) is true''
    or ``\(p\) is false''.

    The two arguments have a common \emph{logical form} of:
    \begin{center}
        If \(p\) or \(q\), then \(r\). \\
        \(\therefore\) If not \(r\), then not \(p\) and not \(q\).
    \end{center}
}


\subsection{Propositions}

\slide{Proposition}{
    \dfn{A \textbf{proposition} is a sentence that is either true or false but not both.}
    \begin{example}
        For example, ``2 + 2 = 4'' and ``2 + 2 = 5'' are both propositions,
        the first because it is true and the second because it is false.
    \end{example}

    \dfn{A \textbf{theorem} is a proposition that can be shown to be true.}
    \begin{example}
        For example, Pythagoras' theorem ``\(a^2 + b^2 = c^2\)'' can be shown to be true
        in a multitude of ways.
    \end{example}
}


\slide{Propositions}{
    Here are some examples of sentences that are \emph{not} propositions:
    \begin{enumerate}
        \item[(1)] \(x^2 + 2 = 11\)
        \item[(2)] \(x + y > 0\)
    \end{enumerate}

    (1) is not a proposition because its truth or falsity depends
    on the value of \(x\). For some values of \(x\), it is true, whereas for other values it is false.

    (2) is not a proposition because its truth or falsity depends
    on the values of \(x\) and \(y\). For instance, when \(x = -1\) and \(y = 2\) it is true, whereas
    when \(x = -1\) and \(y = 1\) it is false.
}


\subsection{Negation, Disjunction, and Conjunction}

\slide{Compound Propositions}{
    We have seen that propositions are sentences that are either true or false but not both.

    Naturally, we would next like to build more complex sentences.
    We now introduce three symbols that are used to build more complicated logical expressions\
    out of simpler ones.
}

\slide{Negation}{
    The symbol \(\neg\) denotes ``not''. Given a proposition \(p\), the sentence \(\neg p\) is read
    ``not \(p\)'' or “it is not the case that \(p\) is true'' or ``\(p\) is false.''

    More formally, it called the \textbf{negation} of \(p\). It can be written in many different ways
    including: \(\sim p\), \(\bar{p}\), \(!p\), or even \(p'\).
}


\slide{Disjunction}{
    The symbol \(\lor\) denotes ``or''. Given propositions \(p\) and \(q\), the sentence \(p \lor q\)
    is read ``\(p\) or \(q\).''

    More formally, \(p \lor q\) is called the \textbf{disjunction} of \(p\) and \(q\). This can also be written
    as: \(p + q\), \(p \ | \ q\), and \(p \ || \ q\).

    Programming languages often use \(p \ || \ q\).
}


\slide{Conjunction}{
    The symbol \(\land\) denotes ``and''. Given propositions \(p\) and \(q\), the sentence \(p \land q\)
    is read ``\(p\) and \(q\).''

    More formally, \(p \land q\) is called the \textbf{conjunction} of \(p\) and \(q\). This can also be written
    as: \(p \times q\), \(p \ \& \ q\), \(p \ \& \& \ q\).

    Programming languages often use \(p \ \& \& \ q\).
}


\slide{Inequalities}{
    The notation for inequalities involves \emph{and} and \emph{or} statements. For instance, if \(x\), \(a\), and \(b\) are
    particular real numbers, then:
    \[\begin{aligned}
    x \le a \quad &\text{means} \quad x < a \quad \text{or} \quad x = a \\
    a \le x \le b \quad &\text{means} \quad a \le x \quad \text{and} \quad x \le b.
    \end{aligned}\]
}


\slide{Order of Operations}{
    In expressions that include the symbol \(\neg\), as well as \(\land\) or \(\lor\), the order of operations
    becomes extremely important.

    The order is as follows:
    \begin{enumerate}
        \item[(1)] \(\neg\) 
        \item[(2)] \(\land\) and \(\lor\)
    \end{enumerate}

    The symbols \(\land\) and \(\lor\) are considered coequal in order of operation, and an expression
    such as \(p \land q \lor r\) is considered ambiguous. This expression must be written as either
    \((p \land q) \lor r\) or \(p \land (q \lor r)\) to have meaning.
}


\subsection{Translating English into Logic}

\slide{But}{
    A variety of words translate into logic as \(\land\), \(\lor\), and \(\neg\).
    
    The word `but' translates the same as `and.' Generally, the word `but' is used in place of `and'
    when the part of the sentence that follows is, in some way, unexpected.
    
    For example, consider the sentence “Jim is tall but he is not heavy.” If we let
    ``Jim is tall'' be the proposition \(t\) and ``he is heavy'' be \(h\), then we can rewrite the
    sentence in logic as \(t \land \neg h\).
}


\slide{Neither-Nor}{
    Another example involves the translating the words `neither-nor' into logic.

    For example, consider the sentence “Neither a borrower nor a lender be.”
    If we let ``Be a borrower'' be the proposition \(b\) and ``be a lender'' be \(d\),
    then we can rewrite the sentence in logic as \(\neg b \land \neg d\).
}


\subsection{Truth Values}

\slide{Truth Value}{
    Recall that a proposition is a sentence that is either true or false but not both.
    If the sentence is true, then its \textbf{truth value} is \textit{true}, otherwise
    it is \textit{false}.

    We now define compound sentences as propositions by specifying their truth values in
    terms of the propositions that compose them. We use letters to
    denote \emph{propositional variables} (or sentential variables), that is, variables
    that represent propositions, just as letters are used to denote numerical variables.

    To do this, we make some definitions about our logical operators.
}


\slide{Negation}{
    \dfn{If \(p\) is a propostional variable, then the \textbf{negation} of \(p\) is ``not \(p\)''
    or ``it is not the case that \(p\)'' and is denoted \(\neg p\). It has the opposite truth value from
    \(p\): if \(p\) is true, \(\neg p\) is false; if \(p\) is false, \(\neg p\) is true.}

    \begin{table}[h]
    \centering
    \renewcommand{\arraystretch}{1.25}  % increases row height
    \setlength{\tabcolsep}{7pt}         % increases column width
    \begin{tabular}{|c|c|}
    \hline
    $p$ & $\neg p$ \\
    \hline
    T & F \\
    \hline
    F & T \\
    \hline
    \end{tabular}
    \caption{Truth Table for the Negation of a Proposition}
    \end{table}
}


\slide{Conjunction}{
    \dfn{If \(p\) and \(q\) are propostional variables, then the \textbf{conjunction} of \(p\) and \(q\) is
    ``\(p\) and \(q\)'' denoted ``\(p \land q\)''. It is true when, and only when, both \(p\) and \(q\)
    are true. If either \(p\) or \(q\) or both are false, then \(p \land q\) is false.}

    \begin{table}[h]
    \centering
    \renewcommand{\arraystretch}{1.25}  % increases row height
    \setlength{\tabcolsep}{7pt}         % increases column width
    \begin{tabular}{|c|c|c|}
    \hline
    $p$ & $q$ & $p \land q$ \\
    \hline
    T & T & T \\
    \hline
    T & F & F \\
    \hline
    F & T & F \\
    \hline
    F & F & F \\
    \hline
    \end{tabular}
    \caption{Truth Table for Conjunction}
    \end{table}
}


\slide{Disjunction}{
    \dfn{If \(p\) and \(q\) are propostional variables, then the \textbf{disjunction} of \(p\) and \(q\) is
    ``\(p\) or \(q\)'' denoted ``\(p \lor q\)''. It is false when, and only when, both \(p\) and \(q\)
    are false. If either \(p\) or \(q\) or both are true, then \(p \lor q\) is true.}

    \begin{table}[h]
    \centering
    \renewcommand{\arraystretch}{1.25} % increases row height
    \setlength{\tabcolsep}{7pt}      % increases column width
    \begin{tabular}{|c|c|c|}
    \hline
    $p$ & $q$ & $p \lor q$ \\
    \hline
    T & T & T \\
    \hline
    T & F & T \\
    \hline
    F & T & T \\
    \hline
    F & F & F \\
    \hline
    \end{tabular}
    \caption{Truth Table for Disjunction}
    \end{table}
}


\slide{Inclusive or Exclusive Or}{
    In ordinary language, `or' is sometimes used in an exclusive sense
    (\(p\) or \(q\) but not both) and sometimes in an inclusive sense (\(p\) or \(q\) or both).
    
    For instance, a waiter who says you may have “coffee, tea, or milk” uses the word or in an
    exclusive sense; extra payment is generally required if you want more than one beverage.
    
    On the other hand, a waiter who offers “cream or sugar” uses the word or in an inclusive sense;
    you are entitled to both cream and sugar if you wish to have them.

    Inclusive `or' is denoted by \(\lor\), while \(\oplus\) is reserved for exclusive `or'.
}


\subsection{Propositional Form}

\slide{Propositional Form}{
    Now that truth values have been assigned to \(\neg p\), \(p \land q\), and \(p \lor q\), consider the question of
    assigning truth values to more complicated expressions such as \(\neg p \lor q\) and \((p \lor q) \land \neg (p \land q)\).

    Such expressions are called propositional forms.

    \dfn{A \textbf{propositional form} is a proposition made up of propositional variables variables (like \(p\), \(q\), and \(r\))
    and logical connectives (like \(\land\), \(\lor\), and \(\neg\)).}

    It is common to refer to a propositional form to as a compound proposition.
}

\slide{Example}{
    Consider the truth table for the compound proposition \((p \land q) \lor \neg r.\)
    
    \begin{center}
    \resizebox{0.5\textwidth}{!}{%
    \begin{tabular}{|c c c|c c|c|}
    \hline
    $p$ & $q$ & $r$ & $p \land q$ & $\neg r$ & $(p \land q)\lor \neg r$ \\
    \hline
    T & T & T & T & F & T \\
    T & T & F & T & T & T \\
    T & F & T & F & F & F \\
    T & F & F & F & T & T \\
    F & T & T & F & F & F \\
    F & T & F & F & T & T \\
    F & F & T & F & F & F \\
    F & F & F & F & T & T \\
    \hline
    \end{tabular}%
    }
    \end{center}
}


\subsection{Logical Equivalence}

\slide{Logical Equivalence}{
    Consider the sentences:
    \begin{enumerate}
        \item[(1)] Dogs bark and cats meow
        \item[(2)] Cats meow and dogs bark 
    \end{enumerate}
    These are two different ways of saying the same thing, the reason has nothing to do with the
    definition of the words. It has to do with the logical form of the statements.

    Any two propositions whose logical forms are related in the same way as (1) and (2) would either
    both be true or both be false.
}


\slide{Logical Equivalence}{
    \dfn{Two compound propositions \(P\) and \(Q\) are called \textbf{logically equivalent} if, and only if, they have
    identical truth values for each possible substitution of propositions for their propositional variables. It is denoted by
    writing: 
    \[P \equiv Q\]}

    How to test whether two compound propositions are logically equivalent?
    \begin{enumerate}
        \item[(1)] Construct a truth table with one column for the truth values of \(P\) and another column for the truth values of \(Q\).
        \item[(2)] If the truth value of \(P\) in each row is the same as the truth value of \(Q\), then \(P\) and \(Q\) are logically equivalent.
    \end{enumerate}
}


\slide{Example 1}{
    Consider the double negation of a proposition \(\neg (\neg p)\). This is logically equivalent to the proposition \(p\).

    This can be proved using the truth table for \(p\), \(\neg p\), and \(\neg (\neg p)\):

    \begin{center}
    \resizebox{0.25\textwidth}{!}{%
    \begin{tabular}{|c|c|c|}
    \hline
    $p$ & $\neg p$ & $\neg (\neg p)$ \\
    \hline
    T & F & T \\
    F & T & F \\
    \hline
    \end{tabular}%
    }
    \end{center}

    The truth value of \(p\) in each row is the same as the truth value of \(\neg (\neg p)\). Therefore, \(p\) and \(\neg (\neg p)\) are logically equivalent.
}


\slide{Example 2}{
    Consider the compound propositions \(\neg (p \land q)\) and \(\neg p \land \neg q\). We can show that
    these two are \emph{not} logically equivalent using their truth tables.

    \begin{center}
    \resizebox{0.75\textwidth}{!}{%
    \begin{tabular}{|c c|c c c|c c c|}
    \hline
    $p$ & $q$ 
    & $\neg p$ & $\neg q$ & $p \land q$ 
    & $\neg (p \land q)$ &  & $\neg p \land \neg q$ \\
    \hline
    T & T & F & F & T & F &  & F \\
    T & F & F & T & F & T & $\neq$ & F \\
    F & T & T & F & F & T & $\neq$ & F \\
    F & F & T & T & F & T &  & T \\
    \hline
    \end{tabular}%
    }
    \end{center}

    Let \(P\) and \(Q\) be the compound propositions \(\neg (p \land q)\) and \(\neg p \land \neg q\), respectively.
    The truth value of \(P\) in each row is \emph{not} the same as the truth value of \(Q\). Therefore, \(P\) and \(Q\) are
    \emph{not} logically equivalent.
}


\subsection{De Morgan's Laws}

\slide{De Morgan's Laws 1}{
    In Example 2, we showed that \(\neg (p \land q)\) and \(\neg p \land \neg q\) are not logically equivalent.
    Naturally, one might ask what is the compound proposition \(\neg (p \land q)\) logically equivalent to.

    To answer the question posed above, consider the proposition “John is tall and Jim is redheaded.” For this
    proposition to be true, both components must be true. So for it to be false, one or both components must be false.
    Thus the negation can be written as “John is not tall or Jim is not redheaded.”
    
    In general, the negation of the conjunction of two statements is logically equivalent to the disjunction of their
    negations.
}


\slide{De Morgan's Laws 1}{
    \begin{center}
    \resizebox{0.75\textwidth}{!}{%
    \begin{tabular}{|c c|c c c|c c|}
    \hline
    $p$ & $q$ 
    & $\neg p$ & $\neg q$ & $p \land q$ 
    & $\neg (p \land q)$ & $\neg p \lor \neg q$ \\
    \hline
    T & T & F & F & T & F & F \\
    T & F & F & T & F & T & T \\
    F & T & T & F & F & T & T \\
    F & F & T & T & F & T & T \\
    \hline
    \end{tabular}%
    }
    \end{center}

    \[\therefore \; \; \neg (p \land q) \equiv \neg p \lor \neg q\]
}


\slide{De Morgan's Laws 2}{
    Similarly, the negation of the disjunction of two propositions is logically equivalent to the conjunction of their negations.

    \begin{center}
    \resizebox{0.75\textwidth}{!}{%
    \begin{tabular}{|c c|c c c|c c|}
    \hline
    $p$ & $q$ 
    & $\neg p$ & $\neg q$ & $p \lor q$ 
    & $\neg (p \lor q)$ & $\neg p \land \neg q$ \\
    \hline
    T & T & F & F & T & F & F \\
    T & F & F & T & T & F & F \\
    F & T & T & F & T & F & F \\
    F & F & T & T & F & T & T \\
    \hline
    \end{tabular}%
    }
    \end{center}

    \[\therefore \; \; \neg (p \lor q) \equiv \neg p \land \neg q\]
}


\slide{Example}{
    Use De Morgan's laws to write the negation of \(-1 < x \le 4\).

    Firstly, the inequality means \(x > -1\) \emph{and} \(x \le 4\).
    Therefore, the negation of this would be \(\neg (x > -1 \land x \le 4)\).
    This is logically equivalent to the disjunction of the negations.

    \[\begin{aligned}
    \neg (x > -1 \land x \le 4) &\equiv \neg (x > -1) \lor \neg (x \le 4) \\
    &\equiv (x \le -1) \lor (x > 4)
    \end{aligned}\]

    \[\therefore \; \; x \in (-\infty, -1] \cup (4, \infty)\]
}


\subsection{Tautologies and Contradictions}

\slide{Tautologies}{
    \dfn{A compound proposition that is always true, no matter what the truth values of the 
    propositional variables that occur in it, is called a \textbf{tautology}.}

    The compound proposition \(p \lor \neg p\) is a tautology. This can be shown using the truth
    table for \(p\), \(\neg p\), and \(p \lor \neg p\):
    \begin{center}
    \resizebox{0.3\textwidth}{!}{%
    \begin{tabular}{|c|c|c|}
    \hline
    $p$ & $\neg p$ & $p \lor \neg p$ \\
    \hline
    T & F & T \\
    F & T & T \\
    \hline
    \end{tabular}%
    }
    \end{center}
}


\slide{Contradictions}{
    \dfn{A compound proposition that is always false, no matter what the truth values of the 
    propositional variables that occur in it, is called a \textbf{contradiction}.}

    The compound proposition \(p \land \neg p\) is a contradiction. This can be shown using the truth
    table for \(p\), \(\neg p\), and \(p \land \neg p\):
    \begin{center}
    \resizebox{0.3\textwidth}{!}{%
    \begin{tabular}{|c|c|c|}
    \hline
    $p$ & $\neg p$ & $p \land \neg p$ \\
    \hline
    T & F & F \\
    F & T & F \\
    \hline
    \end{tabular}%
    }
    \end{center}
}


\slide{Example}{
    If \textbf{t} is a tautology and \textbf{c} is a contradiction, it can be shown that \(p \land t \equiv p\)
    and \(p \land c \equiv c\).

    \begin{center}
    \resizebox{0.5\textwidth}{!}{%
    \begin{tabular}{|c|c|c|c|c|c|}
    \hline
    $p$ & $t$ & $p \wedge t$ & $p$ & $c$ & $p \wedge c$ \\ \hline
    T & T & T & T & F & F \\ \hline
    F & T & F & F & F & F \\ \hline
    \end{tabular}%
    }
    \end{center}

    As evident, \(p \land t \equiv p\) and \(p \land c \equiv c\).
}


\subsection{Logical Equivalence Theorems}

\slide{Theorems}{
    Given any statement variables \(p\), \(q\), and \(r\), a tautology \(t\) and a contradiction \(c\), 
    the following logical equivalences hold.

    \thm{
    \textbf{Commutative Laws}
    \[p \land q \equiv q \land p\]
    \[p \lor q \equiv q \lor p \]}

    \thm{
    \textbf{Associative Laws}
    \[(p \land q) \land r \equiv p \land (q \land r)\]
    \[(p \lor q) \lor r \equiv p \lor (q \lor r)\]}
}


\slide{Theorems}{
    \thm{
    \textbf{Distributive Laws}
    \[p \land (q \lor r) \equiv (p \land q) \lor (p \land r)\]
    \[p \lor (q \land r) \equiv (p \lor q) \land (p \lor r)\]}

    \thm{
    \textbf{Identity Laws}
    \[p \land t \equiv p\]
    \[p \lor c \equiv p\]}
}


\slide{Theorems}{
    \thm{
    \textbf{Negation Laws}
    \[p \lor \neg p \equiv t\]
    \[p \land \neg p \equiv c\]}

    \thm{
    \textbf{Double Negative Law}
    \[\neg (\neg p) \equiv p\]}

    \thm{
    \textbf{Idempotent Laws}
    \[p \land p \equiv p\]
    \[p \lor p \equiv p\]}
}


\slide{Theorems}{
    \thm{
    \textbf{Universal Bound Laws}
    \[p \lor t \equiv t\]
    \[p \land c \equiv c\]}

    \thm{
    \textbf{De Morgan's Laws}
    \[\neg (p \land q) \equiv \neg p \lor \neg q\]
    \[\neg (p \lor q) \equiv \neg p \land \neg q\]}
}


\slide{Theorems}{
    \thm{
    \textbf{Absorption Laws}
    \[p \land (p \lor q) \equiv p\]
    \[p \lor (p \land q) \equiv p\]}

    \thm{
    \textbf{Negations of Tautologies and Contradictions}
    \[\neg t \equiv c\]
    \[\neg c \equiv t\]}
}